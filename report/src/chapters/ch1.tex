\chapter{Introduction}

In this work, we will explore a model able tu simulate the glitches of a pulsar.

Radio pulsar glitches appear to follow an avalanche-like pattern\cite{Melatos_2008}, where the star's superfluid core transfers angular momentum to its solid crust through a series of connected, threshold-triggered events. This process maintains the system in a self-organized critical state.
Analysis of the time intervals between glitches shows an exponential distribution pattern in seven out of nine well-observed pulsars, after accounting for observational constraints on minimum waiting times. This distribution aligns with what we would expect from a constant-rate Poisson process.

A recent study \cite{10.1093/mnras/staa935} proposed a microphysics-agnostic meta-model where internal stress accumulates as a Brownian process between glitches, with glitches triggered when a critical threshold is reached. This model makes specific predictions about glitch statistics, including a Spearman correlation coefficient > 0.25 between glitch size and waiting time. The model's predictions were tested against six pulsars with extensive glitch records, with varying degrees of consistency.
